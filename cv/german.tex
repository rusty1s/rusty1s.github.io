\documentclass[letterpaper,11pt]{article}
\newlength{\outerbordwidth}

\usepackage[a4paper,left=3cm,right=3cm,bottom=2cm,top=2.5cm]{geometry}

\usepackage{style}
\pagestyle{empty}
\raggedbottom
\raggedright
\usepackage[svgnames]{xcolor}
\usepackage{framed}
\usepackage{tocloft}
\usepackage{etoolbox}
\usepackage{tabularx}
\robustify\cftdotfill
\usepackage{hyperref}       % hyperlinks

\usepackage{fontawesome}

\setlength{\outerbordwidth}{3pt}  % Width of border outside of title bars
\definecolor{shadecolor}{gray}{0.75}  % Outer background color of title bars (0 = black, 1 = white)
\definecolor{shadecolorB}{gray}{0.93}  % Inner background color of title bars

\newcommand{\resheading}[1]{
  \parbox{\textwidth}{\setlength{\FrameSep}{\outerbordwidth}
    \begin{shaded}
\setlength{\fboxsep}{0pt}\framebox[\textwidth][l]{\setlength{\fboxsep}{4pt}\fcolorbox{shadecolorB}{shadecolorB}{\textbf{\sffamily{\mbox{~}\makebox[5.662in][l]{\large #1} \vphantom{p\^{E}}}}}}
    \end{shaded}
  }\vspace{-5pt}
}

\newcommand{\ressubheading}[2]{
\begin{tabularx}{\textwidth}{Xr}
  #1 & #2
\end{tabularx}
}

\begin{document}

\begin{tabularx}{\textwidth}{Xr}
  \textbf{\huge Matthias Fey}\\
  \vspace{0.1cm}
  \faMapMarker~Von-der-Recke-Str. 9, 44137 Dortmund\\
  \faEnvelope~\url{matthias.fey@tu-dortmund.de}\\
  % \vspace{0.05cm}
\end{tabularx}

\resheading{Persönliches}

\begin{tabularx}{\textwidth}{XXX}
  \faBirthdayCake~02.06.1990 & \faMapMarker~Unna & \faMale~ledig
\end{tabularx}

\resheading{Studium}

\ressubheading{\textbf{Doktorand} an der TU Dortmund, Fakultät Informatik}{\faCalendar~\emph{2017 - heute}}
\begin{itemize}
  \item Betreuerin: Prof.\ Dr.\ Petra Mutzel
  \item Arbeitstitel: \emph{Geometric Deep Learning: Algorithms \& Applications}
\end{itemize}
\ressubheading{\textbf{Master of Science} an der TU Dortmund, Fakultät Informatik}{\faCalendar~\emph{2013 - 2017}}
\ressubheading{\textbf{Bachelor of Science} an der TU Dortmund, Fakultät Informatik}{\faCalendar~\emph{2010 - 2013}}

\resheading{Schulausbildung}

\ressubheading{\textbf{Abitur} am Pestalozzi Gymnasium Unna}{\faCalendar~\emph{2000 - 2009}}
\ressubheading{Osterfeldschule Unna}{\faCalendar~\emph{1996 - 2000}}

\resheading{Tätigkeiten}

\ressubheading{\textbf{Software-Entwickler} bei Comline, Dortmund (Werkstudent)}{\faCalendar~\emph{2013 - 2017}}
\ressubheading{\textbf{Zivildienstleistender} im Katharinen Hospital Unna}{\faCalendar~\emph{2009 - 2010}}

\resheading{Veröffentlichungen}

\ressubheading{Jan Eric Lenssen, \textbf{Matthias Fey}, Pascal Libuschewski. Group Equivariant Capsule Networks. \textit{Eingereicht.}}{\faCalendar~\emph{2018}}

\ressubheading{Nils M. Kriege, \textbf{Matthias Fey}, Denis Fisseler, Petra Mutzel, Frank Weichert. Recognizing Cuneiform Signs Using Graph Based Methods. In \textit{International Workshop on Cost-Sensitive Learning (COST), SIAM International Conference on Data Mining (SDM).}}{\faCalendar~\emph{2018}}

\ressubheading{\textbf{Matthias Fey}, Jan Eric Lenssen, Frank Weichert, Heinrich Müller. SplineCNN: Fast Geometric Deep Learning with Continuous B-Spline Kernels. In \textit{IEEE Conference on Computer Vision and Pattern Recognition (CVPR).}}{\faCalendar~\emph{2018}}

\ressubheading{Christian Eichhorn, \textbf{Matthias Fey}, Gabriele Kern-Isberner. CP- and OCF-networks – A Comparison. In \textit{Fuzzy Sets and Systems, Volume 298: Special Issue on Graded Logical Approaches and Their Applications.}}{\faCalendar~\emph{2016}}

\resheading{Sprachen}

\begin{tabularx}{\textwidth}{XX}
  \faCircle\,\faCircle\,\faCircle\,\faCircle\,\faCircle~Deutsch
  &
  \faCircle\,\faCircle\,\faCircle\,\faCircle\,\textcolor{lightgray}{\faCircle}~Englisch
\end{tabularx}

\end{document}
