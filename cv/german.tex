\documentclass[letterpaper,11pt]{article}
\newlength{\outerbordwidth}

\usepackage[a4paper,left=2.5cm,right=2.5cm,bottom=2cm,top=2cm]{geometry}

\usepackage{style}
\pagestyle{empty}
\raggedbottom
\raggedright
\usepackage[svgnames]{xcolor}
\usepackage{framed}
\usepackage{tocloft}
\usepackage{etoolbox}
\usepackage{tabularx}
\robustify\cftdotfill
\usepackage{hyperref}       % hyperlinks

\usepackage{fontawesome}

\title{Lebenslauf}

\author{Matthias Fey\\
Von-der-Recke-Str. 9\\
\faBirthdayCake 44137 Dortmund\\
\texttt{matthias.fey@tu-dortmund.de}
}

\setlength{\outerbordwidth}{3pt}  % Width of border outside of title bars
\definecolor{shadecolor}{gray}{0.75}  % Outer background color of title bars (0 = black, 1 = white)
\definecolor{shadecolorB}{gray}{0.93}  % Inner background color of title bars

\newcommand{\resheading}[1]{
  \parbox{\textwidth}{\setlength{\FrameSep}{\outerbordwidth}
    \begin{shaded}
\setlength{\fboxsep}{0pt}\framebox[\textwidth][l]{\setlength{\fboxsep}{4pt}\fcolorbox{shadecolorB}{shadecolorB}{\textbf{\sffamily{\mbox{~}\makebox[6.762in][l]{\large #1} \vphantom{p\^{E}}}}}}
    \end{shaded}
  }\vspace{-5pt}
}

\newcommand{\ressubheading}[2]{
\begin{tabularx}{\textwidth}{Xr}
  #1 & #2 \\
\end{tabularx}
}

\begin{document}

\begin{tabularx}{\textwidth}{Xr}
  \textbf{\huge Matthias Fey}\\
  \vspace{0.1cm}
  \faMapMarker~Von-der-Recke-Str. 9, 44137 Dortmund\\
  \faEnvelope~\url{matthias.fey@tu-dortmund.de}\\
  \vspace{0.1cm}
\end{tabularx}

\resheading{Persönliches}

\begin{tabularx}{\textwidth}{XXX}
  \faBirthdayCake~02.06.1990 & \faMapMarker~Unna & \faMale~ledig\\
\end{tabularx}

\resheading{Studium}

\ressubheading{\textbf{Doktorand} an der TU Dortmund, Fakultät Informatik}{\faCalendar~\emph{2017 - heute}}
\begin{itemize}
  \item Betreuerin: Prof.\ Dr.\ Petra Mutzel
  \item Arbeitstitel: \emph{Geometric Deep Learning: Algorithms \& Applications}
\end{itemize}
\ressubheading{\textbf{Master of Science} an der TU Dortmund, Fakultät Informatik}{\faCalendar~\emph{2010 - 2013}}
\ressubheading{\textbf{Bachelor of Science} an der TU Dortmund, Fakultät Informatik}{\faCalendar~\emph{2013 - 2017}}

\resheading{Schulausbildung}

\ressubheading{\textbf{Abitur} am Pestalozzi Gymnasium Unna}{\faCalendar~\emph{2000 - 2009}}
\ressubheading{Osterfeldschule Unna}{\faCalendar~\emph{1996 - 2000}}

\resheading{Taetigkeiten}

\ressubheading{\textbf{Software-Entwickler} bei Comline, Dortmund (Werkstudent)}{\faCalendar~\emph{2013 - 2017}}
\ressubheading{\textbf{Zivildienstleistender} im Katharinen Hospital Unna}{\faCalendar~\emph{2009 - 2010}}

\resheading{Veroeffentlichungen}

\begin{tabular}{p{3cm}p{13cm}}
2016 & Christian Eichhorn, Matthias Fey, Gabriele Kern-Isberner\\
     & \textit{CP- and OCF-networks – A Comparison}\\
     & Fuzzy Sets and Systems, Volume 298: Special Issue on Graded Logical Approaches and Their Applications\\
2018 & Matthias Fey, Jan Eric Lenssen, Frank Weichert, Heinrich Müller\\
     & \textit{SplineCNN: Fast Geometric Deep Learning with Continuous B-Spline Kernels}\\
     & IEEE Conference on Computer Vision and Pattern Recognition (CVPR)\\
2018 & Nils M. Kriege, Matthias Fey, Denis Fisseler, Petra Mutzel, Frank Weichert\\
     & \textit{Recognizing Cuneiform Signs Using Graph Based Methods}\\
     & International Workshop on Cost-Sensitive Learning (COST), SIAM International Conference on Data Mining (SDM)\\
\end{tabular}


% \section*{Persönliche Daten}

% \begin{tabular}{p{3cm}p{13cm}}
% Name          & Matthias Fey\\
% Geburtsdatum  & 02.06.1990\\
% Familienstand & ledig\\
% \end{tabular}

% % \section*{Schulbildung}

% % \begin{tabular}{p{3cm}p{13cm}}
% % 1996 – 2000 & Osterfeldschule Unna\\
% % 2000 – 2009 & Pestalozzi-Gymnasium Unna\\
% %             & Abschluss: Abitur (2,2)\\
% % \end{tabular}

% % \section*{Studium}

% % \begin{tabular}{p{3cm}p{13cm}}
% % Oktober 2010 – Dezember 2013  & B.Sc.\ Informatik, TU Dortmund\\
% % Dezember 2013 – September 2017 & M.Sc.\ Informatik, TU Dortmund\\
% % Oktober 2017 - heute & Doktorand 
% % \end{tabular}

% % \subsection*{Tätigkeiten}

% % \begin{tabular}{p{3cm}p{13cm}}
% % 2009 – 2010 & Zivildienst, Katharinen Hospital Unna\\
% % 2013 – 2017 & Software-Entwickler (Werkstudent), Comline\\
% % 2017 - heute & Wissenschaftlicher Mitarbeiter, TU Dortmund\\
% % \end{tabular}

% % \subsection*{Veröffentlichungen}

% % \begin{tabular}{p{3cm}p{13cm}}
% % 2016 & Christian Eichhorn, Matthias Fey, Gabriele Kern-Isberner\\
% %      & \textit{CP- and OCF-networks – A Comparison}\\
% %      & Fuzzy Sets and Systems, Volume 298: Special Issue on Graded Logical Approaches and Their Applications\\
% % 2018 & Matthias Fey, Jan Eric Lenssen, Frank Weichert, Heinrich Müller\\
% %      & \textit{SplineCNN: Fast Geometric Deep Learning with Continuous B-Spline Kernels}\\
% %      & IEEE Conference on Computer Vision and Pattern Recognition (CVPR)\\
% % 2018 & Nils M. Kriege, Matthias Fey, Denis Fisseler, Petra Mutzel, Frank Weichert\\
% %      & \textit{Recognizing Cuneiform Signs Using Graph Based Methods}\\
% %      & International Workshop on Cost-Sensitive Learning (COST), SIAM International Conference on Data Mining (SDM)\\
% % \end{tabular}

% % \subsection*{Lehre}

% % \begin{tabular}{p{3cm}p{13cm}}
% % 2017 - 2018 & Prof.~Dr.~Müller, Informatik VII, TU Dortmund\\
% %             & \textit{Mensch-Maschine-Interaktion}\\
% % \end{tabular}

% % \subsection*{Interne Arbeiten}

% % \begin{tabular}{p{3cm}p{13cm}}
% % 2013 & Bachelor-Arbeit: Prof.~Dr.~Kern-Isberner, Informatik I, TU Dortmund\\
% %      & \textit{Qualitative Semantiken für DAGs – ein Vergleich von OCF- und CP-Netzwerken}\\
% % 2017 & Master-Arbeit: Prof.~Dr.~Müller, Informatik VII, TU Dortmund\\
% %      & \textit{Convolutional Neural Networks auf Graphrepräsentationen von Bildern}\\
% % \end{tabular}

% % \subsection*{Fortbildungen}

% % \begin{tabular}{p{3cm}p{13cm}}
% % 2013 & SCRUM-Workshop, MSG Systems und TU Dortmund\\
% % \end{tabular}

% % \subsection*{Kenntnisse}

% % \begin{tabular}{p{3cm}p{13cm}}
% % Fremdsprachen     & Englisch\\
% % Betriebssysteme   & macOS, Linux, Windows\\
% % Programmierspr.   & Python, JavaScript, Java, C/C++, Swift\\
% % Auszeichnungsspr. & HTML, CSS\\
% % Frameworks        & TensorFlow, React, Ruby on Rails, Spring, jQuery\\
% % Sonstiges         & Photoshop CC, Latex, SQL\\
% % \end{tabular}
\end{document}
