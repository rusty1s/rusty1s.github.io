\documentclass[10pt]{scrartcl}

\usepackage[a4paper,left=2.5cm,right=2.5cm,bottom=2cm,top=2cm]{geometry}

\usepackage{lmodern}
\usepackage[english,main=ngerman]{babel}
\usepackage[utf8]{inputenc}
\usepackage[T1]{fontenc}

\begin{document}

\pagenumbering{gobble}

\section*{Lebenslauf}

\subsection*{Persönliche Daten}

\begin{tabular}{p{3cm}p{13cm}}
Name          & Matthias Fey\\
Geburtsdatum  & 02.06.1990\\
Familienstand & ledig\\
\end{tabular}

\subsection*{Schulbildung}

\begin{tabular}{p{3cm}p{13cm}}
1996 – 2000 & Osterfeldschule Unna\\
2000 – 2009 & Pestalozzi-Gymnasium Unna\\
            & Abschluss: Abitur\\
\end{tabular}

\subsection*{Studium}

\begin{tabular}{p{3cm}p{13cm}}
2010 – 2013  & B.Sc.\ Informatik, TU Dortmund\\
2013 – heute & M.Sc.\ Informatik, TU Dortmund\\
\end{tabular}

\subsection*{Tätigkeiten}

\begin{tabular}{p{3cm}p{13cm}}
2009 – 2010 & Zivildienst, Katharinen Hospital Unna\\
2013 – 2017 & Software-Entwickler (Werkstudent), Comline\\
\end{tabular}

\subsection*{Veröffentlichungen}

\begin{tabular}{p{3cm}p{13cm}}
2016 & CP- and OCF-networks – a comparison\\
     & Christian Eichhorn, Matthias Fey, Gabriele Kern-Isberner\\
     & \textit{Fuzzy Sets and Systems, Volume 298: Special Issue on Graded Logical Approaches and Their Applications}\\
\end{tabular}

\subsection*{Interne Arbeiten}

\begin{tabular}{p{3cm}p{13cm}}
2013 & Bachelor-Arbeit bei Prof.~Dr.~Kern-Isberner, Informatik I, TU Dortmund\\
     & \textit{Qualitative Semantiken für DAGs – ein Vergleich von OCF- und CP-Netzwerken}\\
2017 & Master-Arbeit bei Prof.~Dr.~Müller, Informatik VII, TU Dortmund\\
     & \textit{Convolutional Neural Networks auf Graphrepräsentationen von Bildern}\\
\end{tabular}

\subsection*{Fortbildungen}

\begin{tabular}{p{3cm}p{13cm}}
2013 & SCRUM-Workshop, MSG Systems und TU Dortmund\\
\end{tabular}

\subsection*{Kenntnisse}

\begin{tabular}{p{3cm}p{13cm}}
Fremdsprachen     & Englisch\\
Betriebssysteme   & macOS, Linux, Windows\\
Programmierspr.   & Python, JavaScript, Java, C/C++, Swift\\
Auszeichnungsspr. & HTML, CSS\\
Frameworks        & TensorFlow, React, Ruby on Rails, Spring, jQuery\\
Sonstiges         & Photoshop CC, Latex, SQL\\
\end{tabular}

\end{document}
